\newpage
\begin{center}
  \textbf{\large АННОТАЦИЯ}
\end{center}

Один из способов косвенного обнаружения темной материи --- детектирование сигнала нейтринными обсерваториями потоков нейтрино, образующихся в результате аннигиляции темной материи, которая захватывается и накапливается  в гравитационном поле небесных объектов, таких как Солнце или Земля. На данный момент этот сигнал не обнаружен, что накладывает ограничения на сечение взаимодействия частицы темной материи и протона. В данной работе мы получим такие ограничения для неупругой темной материи, состоящей из основного и возбужденного состояния, для которой ограничения на сечения ослаблены по сравнению с упругой темной материей. 

Для вычисления темпов аннигиляции темной материи необходимо найти распределение темной материи внутри небесного объекта на данный момент, для чего мы будем численно решать уравнение термализации. При этом мы будем вычислять распределение частиц темной материи и исследовать аннигиляционные потоки, поскольку неупругая темная материя при большой разнице масс основного и возбужденного состояния не успевает прийти в термальному равновесию, что приводит к еще большему ослаблению ограничений на сечение.

\onehalfspacing
\setcounter{page}{2}

\newpage
\renewcommand{\contentsname}{\centerline{\large СОДЕРЖАНИЕ}}
\tableofcontents

\newpage
\begin{center}
  \textbf{\large ВВЕДЕНИЕ}
\end{center}
\addcontentsline{toc}{chapter}{ВВЕДЕНИЕ}


Одна из основных проблем современной космологии --- это проблема темной материи.  Темная материя --- это часть нерелятивисткой материи, находящейся во Вселенной, которая не наблюдается напрямую, однако участвует в расширении Вселенной, образовании структур. 


Наблюдения за астрономическими объектами и исследования анизотропии реликтового излучения позволяет найти основные параметры космологических моделей, такие как доля нерелятивистского вещества и барионной материи. Для современных космологических моделей доля вещества в настоящий момент составляет $\Omega_M = 0.2-0.3$, при этом доля барионной материи составляет всего $0.03-0.05$ \cite{Cao_2023}. Это означает, что большая часть вещества во Вселенной остается не объясненной на сегодняшний день наблюдаемой материей.

На наличие темной материи также указывают наблюдения скоростей звезд внутри галактик. Исследование скоростей звезд внутри галактик позволяет найти распределение массы и плотность материи \cite{Radial_velocity_measurements}, \cite{Angular_Velocity}. Отношение измеренной таким образом гравитационной массы и массы наблюдаемого светимого вещества оказывается значительно больше единицы. Аналогичные исследования позволяют найти локальную плотность темной матери в Солнечной системе. Эта плотность равна $\rho_{DM} = 0.2 - 0.4, \text{ГэВ} \cdot \text{см}^{-3}$ \cite{palau2022oblateness}. В данной работе мы будем использовать значение  $\rho_{DM} = 0.4, \text{ГэВ} \cdot  \text{см}^{-3}$.


Существуют различные способы объяснить темную материю. Так, в качестве темной материи могут быть массивные астрономические компактные объекты (первичные черные дыры с массой порядка $10-100M_{\odot}$). Такие объекты обнаруживаются с помощью измерения светимости и гравитационного линзирования, и на сегодняшний день наблюдения дают ограничения на их долю в массе нерелятивисткого вещества в районе $0.15-0.3$ \cite{Zumalac_rregui_2018}, \cite{Blaineau_2022}.
Также существуют различные модификации теории гравитации, которые могут объяснить кривые вращения или вклад материи в метрики без включения в модель новых частиц \cite{1984ApJ...286....7B}.


Наиболее распространенные модели темной материи предполагают наличие новых частиц вне стандартной модели, которые находятся в активном поиске. В качестве кандидатов рассматривают, например, майорановские стерильные нейтрино \cite{Boyarsky_2019}, наличие которых может указать регистрация двойного безнейтринного $\beta$-распада или спектральных линий фотонов, возникающих при их распаде. 

Также, темной материей могут быть аксионы, призванные решить проблему сильных CP нарушений, которые могут осциллировать в фотоны в сильных электромагнитных полях \cite{adams2023axion}. Частицы темной материи появляются и в суперсимметричных расширениях стандартной	 модели \cite{berezinsky1996dark}, так как из-за сохранения R-четности легчайшая частица-суперпартнер становится стабильной и может быть основой для массивных слабовзаимодействующих частиц, о которых будет идти речь далее.

Массивные слабовзаимодействующие частицы (WIMP) --- это частицы темной материи в широком диапазоне масс ($\text{МэВ}-\text{ТэВ}$). Предполагается, что такие частицы находились в термальном равновесии с остальной материей на ранних этапах эволюции Вселенной. Затем, при расширении Вселенной, когда темп аннигиляции становится меньше темпа расширения (постоянная Хаббла на соответствующий момент времени), эти частицы замораживаются, будучи нерелятивистскими\cite{Kolb:1990vq}. Соответствующая температура определяется соотношением: 
\begin{equation}
	x_f = \cfrac{m_{\chi} }{T_f} = \ln {\left(\cfrac{0.038 g_{\chi} M_{pl} m_{\chi}  \average{\sigma_{ann} v}}
		{\sqrt{g_* x_f}}\right)}
\end{equation}
где $g_{\chi}$ и $g_*$ --- степени свободы темной материи и релятивистского вещества, $M_{pl}$ --- масса Планка, $\average{\sigma_{ann} v} = \sigma_0$ --- среднее сечение аннигиляции умноженное на скорость. И доля темной материи, состоящей из этих частиц, равна: 
\begin{equation}
	\Omega{\chi} = \cfrac{\sn{1.9}{-27} x_f}{\sqrt{g_*} \sigma_0} \frac{cm^3}{s}
\end{equation}
При разумных параметрах ($x_f \approx 20, g_* \approx 80$), для объяснения сегодняшней плотности темной материи данным механизмом необходимо, чтобы $\sigma_0 \approx 10^{26} \text{см}^3\text{с}^{-1}$, что по порядку величины близко к слабым взаимодействиям.

Темная материя, находящаяся в галактическом гало, может быть обнаружена прямыми методами в низкофоновых экспериментах. Такой способ основан на детектировании отдачи при взаимодействии частиц темной материи с ядром \cite{Schumann_2019}. Наиболее известные эксперименты --- DAMA/LIBRA, COSINE-100, XENON100, XENON1T, CDMS, использующие в качестве мишени NaI, Xe, Ge. На данный момент эти эксперименты не обнаружили значительного превышения сигнала над фоном, кроме DAMA/LIBRA, регистрирующий сигнал годовых модуляций, который свидетельствует о наличие темной материи \cite{Bernabei_2018}. Однако, эксперимент COSINE-100, имеющий ту же мишень (NaI), не подтвердил результаты \cite{Adhikari_2022}.

В данной работе рассматривается косвенный метод обнаружения темной материи, основанный на детектировании аннигиляционных потоков частиц, захваченных небесными телами. Частицы темной материи, взаимодействуя с веществом, сосредоточенном в астрономических объектах, передают им часть кинетической энергии. Это приводит к захвату частиц в гравитационном потенциале небесного тела. В результате темная материя накапливается, что приводит к значительному усилению темпа аннигиляции. Таким образом создаются потоки нейтрино, которые возможно зарегистрировать в нейтринных обсерваториях  IceCube \cite{Aartsen_2017}, SuperKamiokande \cite{kamiokandecollaboration2015search}, ANTARES \cite{ADRIANMARTINEZ201669}. Для захвата частиц темной материи рассматриваются, как правило, Солнце \cite{1985ApJ...296..679P} или Земля \cite{1987ApJ...321..571G}. Для более легких частиц темной материи рассматривается Юпитер \cite{French_2022}, так как испарение на Юпитере значительно ниже). Отсутствие нейтринного сигнала дает ограничение на сечение взаимодействия с протоном $\sigma_{\chi p}$.

\begin{figure}[htb]
	\begin{center}
		\includegraphics[scale=0.9]{images/SK_graphs.pdf}
		\caption{Ограничения на сечение взаимодействия из разных экспериментов с протоном $\sigma_{\chi p}$. Слева --- для спин независимых взаимодействий, справа для спин зависимых взимодействий \cite{kamiokandecollaboration2015search}.}
	\end{center}
\end{figure}

Поскольку на данный момент ни прямыми, ни косвенными методами не удалось обнаружить частицы темной материи, то создаются более сложные модели частиц темной материи.
В данной работе рассматривается двухкомпонентная неупругая темная материя, частица которой имеет основное $\chi$ и возбужденное состояние $\chi^*$ с массами $m_{\chi}$ и $m_{\chi}+\delta$, соответственно. Изначально такая модификация частиц темной материи была предложена для объяснения расхождений между экспериментом DAMA/LIBRA и CDMS \cite{PhysRevD.64.043502}, поскольку от массы мишени зависит, будет ли преодолен энергетический порог $\delta$. Хотя результаты DAMA/LIBRA не смогли воспроизвестись на COSINE-100, такие модели могут ослабить ограничения на сечения, и поэтому представляют интерес.

Важным отличием неупругой темной материи является нетривиальная термализация. Если в упругом случае частицы приходят в больцмановское равновесие, то неупругая темная материя может не успеть прийти в термальное равновесие с небесным телом \cite{Blennow_2018}. Поэтому термализация требует более детального анализа.

Целью данной работы является получение ограничения на сечения рассеяния частицы темной материи с учетом процессов термализации, исходя из данных нейтринных обсерваторий.

