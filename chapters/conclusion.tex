\clearpage
\newpage
\begin{center}
	\textbf{\large ЗАКЛЮЧЕНИЕ}
\end{center}

\refstepcounter{chapter}
\addcontentsline{toc}{chapter}{ЗАКЛЮЧЕНИЕ}

В результате проведенного исследования получены ограничения на сечение взаимодействия неупругой темной материи (имеющей основное и возбужденное состояние с разницей масс в $\delta$) с протоном. 

Расчетные данные для исследования были получены методом детектирования аннигиляционных потоков от частиц, захваченных гравитационным полем небесных тел. Для этого мы рассматривали эволюцию распределения числа частиц темной материи в плоскости энергия-импульс. Поскольку распределение при термализации не успевает дойти до больцмановсого равновесия при больших $\delta$, темп аннигиляции будет меньше, чем при термальном равновесии, а значит и ограничения на сечение $\sigma_{\chi p}$  более слабые. Более того, захват неупругой темной материи кинематически ослаблен по сравнению с упругой, что также ослабляет существующие ограничения на сечения темной материи и протона.

В данной работе мы использовали спин-независимую модель взаимодействия темной материи с веществом, в котором упругое взаимодействие полностью подавлено. Данное исследование можно продолжить, рассматривая более сложные варианты взаимодействия.

Также мы изучали эволюцию тяжелых частиц темной материи, для которых испарение не играет значительно роли. При этом большой интерес представляет термализация с учетом испарения для более легких частиц. Более того, для сверхлегких частиц темной материи, которые на Земле и на Солнце плохо аккумулируются из-за испарения, интерес представляет Юпитер, на котором возможна нетривиальная эволюция частиц неупругой темной материи за счет возбуждения и ионизации атомов водорода.