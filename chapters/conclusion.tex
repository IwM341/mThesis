%\clearpage
%\pagebreak
\newpage
\begin{center}
	\textbf{\large ЗАКЛЮЧЕНИЕ}
\end{center}

\refstepcounter{chapter}
\addcontentsline{toc}{chapter}{ЗАКЛЮЧЕНИЕ}

В данной работе получены ограничения на сечение взаимодействия неупругой темной материи (имеющей основное и возбужденное состояние с разницей масс $\delta$) с протоном исходя из отсутствия нейтринного сигнала от аннигиляции темной материи внутри небесных тел.
Для этого мы изучим эволюцию распределения числа частиц темной материи в плоскости энергия-импульс. Поскольку распределение при этой эволюции не успевает дойти до больцмановского равновесия при больших $\delta$, темп аннигиляции оказывается меньше, чем при термальном равновесии, а значит, ограничения на сечение $\sigma_{\chi p}$  становятся слабее. Более того, захват неупругой темной материи кинематически подавлен по сравнению с упругой, что также ослабляет существующие ограничения на сечения темной материи и протона.

В данной работе мы рассматривали только спин-независимое взаимодействие темной материи с веществом, в котором упругое взаимодействие полностью подавлено. Для обобщения, интересно рассмотреть другие варианты потенциалов взаимодействия \cite{Fitzpatrick_2013}.

В данной мы изучали эволюцию тяжелых частиц темной материи, для которых испарение не играет значительной роли. Однако, большой интерес представляет термализация более легких частиц, где заметную роль играет испарение. Более того, для частиц темной материи меньше ГэВ, которые на Земле и на Солнце плохо аккумулируются из-за испарения, интерес представляет Юпитер, на котором возможна нетривиальная эволюция частиц неупругой темной материи за счет возбуждения и ионизации атомов водорода.